\documentclass[a4paper, 10pt]{article}
% Packages
\usepackage[utf8]{inputenc}  % For Unicode support
\usepackage{amsmath}         % For math symbols
\usepackage{graphicx}        % For including images
\usepackage{hyperref}        % For hyperlinks
\usepackage{graphicx}        % For image rendering
\usepackage{listings}        % For code blocks
\usepackage{sectsty}         % For section font formatting
\usepackage[a4paper, total={7in,9.5in}]{geometry} % For Margins
\usepackage{xcolor}
\usepackage{hyperref}
\usepackage{mathptmx}

% Package Initialization
\graphicspath{{./images/}}

% Formatting
\sectionfont{\fontsize{16}{14}\selectfont}
\hypersetup{
    colorlinks=true,
    linkcolor=blue,
    filecolor=magenta,
    urlcolor=cyan,
}

% Title, author, date
\title{Tao Te Ching}
\author{ksolomon}
\date{\today}

\begin{document}

\maketitle

\begin{abstract}
	Personal notes while reading Tao Te Ching.
\end{abstract}


%%%%%%%%%%%%%%%%%%%%%%%%%%%%%%%%%%%%%%%%%%%%%%%%%%%%%%%%%%%%
% Part 1
%%%%%%%%%%%%%%%%%%%%%%%%%%%%%%%%%%%%%%%%%%%%%%%%%%%%%%%%%%%%

\section{Part 1}
\subsection{}
A Tao is a path. Any path that people walk will change over time. Attempts to control the flow of time will be futile, not due to determinism, but due to circumstance. \textit{"Always without desire we must be found...if desire always within us be, its outer fringe is all we shall see." "Where the Mystery is the deepest is the gate of all that is subtle and wonderful."} This proverb really drives the concept that getting out of your comfort zone and allowing for things to happen naturally is where beauty in life lies. I've been trying to force a specific outcome by *working hard*, however this is simply causing me pain whenever the outcome doesn't go according to plan. If instead I do my best and tolerate circumstance to cause a change of plans I'll be far more satisfied.
\subsection{}
Everything is relative. Contrast creates perspective. The wise manage affairs without action and convey instructions without speech, rather they let things unfold naturally. \textit{"The work is done, but how no one can see; 'tis this that makes power not cease to be."} This proverb seems to speak to having internal sense of motivation, but allowing for external locus of control.
\subsection{}
How to keep a population happy: \textit{"the sage...empties their minds, fills their bellies, weakens their wills, and strengthens their bones."} Basically, keep your population thoughtless and without care. \textit{"When there is abstinence from action, good order is universal."} I wonder how much of this is just a sleight on humanity as a whole?
\subsection{}
\textit{"The Tao is the emptiness of a vessel...be on guard against fulness."}
\subsection{}
\begin{quote}
	\centering
	\textit{"Heaven and earth do not act from any wish to be benevolent; they deal with all things as the dogs of grass"}
\end{quote}
Be unbiased.
\begin{quote}
	\centering
	\textit{"Much speech to swift exhaustion lead we see; Your inner being guard, and keep it free."}
\end{quote}
Listen more than you speak.
\subsection{}
The powers that remain unbroken in the face of adversity are those which persist.
\subsection{}
Those who live long lives do not live for themselves. By putting others before yourself you'll find success.
\begin{quote}
	\centering
	\textit{"Is it not because he has no personal and private ends, that therefore such ends are realized?"}
\end{quote}
\subsection{}
Glory befalls those who are like water: benefitting all things. When someone benefits all things and does not complain, no one can find fault with them.
\subsection{}
All things are only good and valuable in moderation. Abundance of praise/honor spawns arrogance which corrupts.
\begin{quote}
	\centering
	\textit{"When the work is done, and one's name has become distinguished, to withdraw into obscurity is the way of Heaven".}
\end{quote}
\end{document}

